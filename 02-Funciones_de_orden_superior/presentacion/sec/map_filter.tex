\section{Estructuras de datos}
\subsection{Listas}
\subsection{Árboles Binarios}
\begin{frame}
  \frametitle{Estructuras de datos}
  \framesubtitle{Definición}
  Estas son las implementaciones de las estructuras
  de datos que vamos a utilizar:
  \begin{itemize}
  \item\textbf{Listas}
    \inputminted[bgcolor=bg]{haskell}{code/List.hs}
  \item\textbf{Árboles Binarios}
    \inputminted[bgcolor=bg]{haskell}{code/BinaryTree.hs}
  \end{itemize}
\end{frame}

\begin{frame}
  \frametitle{Estructuras de datos} \framesubtitle{Representación
    gráfica} Así es como representamos gráficamente la lista [1,2,3] en Haskell:

  \begin{center}                %Esto no habría sido posible sin Pablo Conde de la Mata
    \begin{tikzpicture} [node distance=1.5cm, every node/.style={font=\sffamily}, align=center]
      \node (raiz)   {$(:)$};
      \node (hijo1) [below of=raiz, xshift=-1cm] {$1$};
      \node (hijo2) [below of=raiz, xshift=1cm] {$(:)$};

      \node (hijo3) [below of=hijo2, xshift=-1cm] {$2$};
      \node (hijo4) [below of=hijo2, xshift=1cm] {$(:)$};

      \node (hijo5) [below of=hijo4, xshift=-1cm] {$3$};
      \node (hijo6) [below of=hijo4, xshift=1cm] {$[]$};


      \draw[->] (raiz) -- (hijo1);
      \draw[->] (raiz) -- (hijo2);
      \draw[->] (hijo2) -- (hijo3);
      \draw[->] (hijo2) -- (hijo4);
      \draw[->] (hijo4) -- (hijo5);
      \draw[->] (hijo4) -- (hijo6);
    \end{tikzpicture}
  \end{center}
  % \input{sec/diagrama_lista}
\end{frame}

\section{Funciones de Orden Superior}
\subsection{map}
\begin{frame}
  \frametitle{Funciones de orden Superior}
  \framesubtitle{\textit{map}}
  La función \textit{map} es una función de orden superior
  que aplica \textit{f} a una estructura de datos.
  \begin{columns}
    \begin{column}{0.5\textwidth}
      Su tipo es:
      \inputminted[bgcolor=bg]{haskell}{code/map.hs}
    \end{column}
    \begin{column}{0.5\textwidth}  %%<--- here
      Su representación gráfica sobre la lista [1,2,3]:
      \begin{center} %Esto no habría sido posible sin Pablo Conde de la Mata
        \begin{tikzpicture} [node distance=1.5cm, every node/.style={font=\sffamily}, align=center]
          \node (raiz)   {$(:)$};
          \node (hijo1) [below of=raiz, xshift=-1cm] {$f\thinspace1$};
          \node (hijo2) [below of=raiz, xshift=1cm] {$(:)$};

          \node (hijo3) [below of=hijo2, xshift=-1cm] {$f\thinspace2$};
          \node (hijo4) [below of=hijo2, xshift=1cm] {$(:)$};

          \node (hijo5) [below of=hijo4, xshift=-1cm] {$f\thinspace3$};
          \node (hijo6) [below of=hijo4, xshift=1cm] {$[]$};


          \draw[->] (raiz) -- (hijo1);
          \draw[->] (raiz) -- (hijo2);
          \draw[->] (hijo2) -- (hijo3);
          \draw[->] (hijo2) -- (hijo4);
          \draw[->] (hijo4) -- (hijo5);
          \draw[->] (hijo4) -- (hijo6);
        \end{tikzpicture}
      \end{center}
    \end{column}
  \end{columns}

\end{frame}

\subsection{filter}
\begin{frame}
  \frametitle{Funciones de orden Superior}
  \framesubtitle{\textit{filter}} La función \textit{filter} es una
  función que recibe un predicado y una lista y devuelve la lista
  filtrada.
  \inputminted[bgcolor=bg]{haskell}{code/filter.hs}
  Ejemplo:
  {\color{white}
    \inputminted[bgcolor=bg]{text}{code/filter.txt}
  }
\end{frame}

\subsection{fold}
\begin{frame}
  \frametitle{Funciones de orden Superior}
  \framesubtitle{\textit{fold}} La función \textit{fold} es una
  función de orden superior que permite compactar una estructura de
  datos a un solo valor.
  % \begin{columns}
  %   \begin{column}{0.5\textwidth}
  \inputminted[bgcolor=bg]{haskell}{code/foldr.hs}
  % \end{column}
  % \begin{column}{0.5\textwidth}
  \begin{center}                %Esto no habría sido posible sin Pablo Conde de la Mata
    \begin{tikzpicture} [node distance=1.5cm, every node/.style={font=\sffamily}, align=center]
      \node (raiz)   {$f$};
      \node (hijo1) [below of=raiz, xshift=-1cm] {$1$};
      \node (hijo2) [below of=raiz, xshift=1cm] {$f$};

      \node (hijo3) [below of=hijo2, xshift=-1cm] {$2$};
      \node (hijo4) [below of=hijo2, xshift=1cm] {$f$};

      \node (hijo5) [below of=hijo4, xshift=-1cm] {$3$};
      \node (hijo6) [below of=hijo4, xshift=1cm] {$acc$};


      \draw[->] (raiz) -- (hijo1);
      \draw[->] (raiz) -- (hijo2);
      \draw[->] (hijo2) -- (hijo3);
      \draw[->] (hijo2) -- (hijo4);
      \draw[->] (hijo4) -- (hijo5);
      \draw[->] (hijo4) -- (hijo6);
    \end{tikzpicture}
  \end{center}
  % \end{column}
  % \end{columns}
\end{frame}