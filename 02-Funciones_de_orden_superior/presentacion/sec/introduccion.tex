\subsection{Curried functions}

\begin{frame}[fragile]
  \frametitle{Introducción}
  \framesubtitle{\textit{Curried functions}}
  Las funciones en Haskell reciben \textbf{un único argumento}.

  {\color{white}
    \inputminted[bgcolor=bg]{text}{code/curr00.txt}
  }

  Si observamos el tipo de max:
  {\color{white}
    \inputminted[bgcolor=bg]{haskell}{code/curr01.hs}
  }
  que es equivalente a:
  {\color{white}
    \inputminted[bgcolor=bg]{haskell}{code/curr02.hs}
  }

\end{frame}

\begin{frame}[fragile]
  \frametitle{Introducción} \framesubtitle{\textit{Curried functions}}
  Las siguientes funciones son equivalentes:
  {\color{white}
    \inputminted[bgcolor=bg]{haskell}{code/curr03.hs}
  }
  Entonces... ¿Qué sucede cuando llamamos a una función con menos
  argumentos de los que deberíamos?
  {\color{white}
    \inputminted[bgcolor=bg]{text}{code/curr04.txt}
  }
\end{frame}